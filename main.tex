\documentclass[pre,12pt]{revtex4-1}
\usepackage{epsfig,graphics,amssymb,amsmath,subeqnarray,setspace,graphicx,amsthm,subfigure, mathrsfs,colortbl,color,bm,fancyhdr}

% Header on each page, with team number and page numbering
\pagestyle{fancy}
\lhead{Team \# 12345} 
\rhead{page \thepage \ of \pageref{LastPage}}
\cfoot{}
\renewcommand{\headrulewidth}{0.4pt}
\renewcommand{\footrulewidth}{0.0pt}

% Make life easier: define some shortcuts!
\def\b{\bm}
\def\e{\epsilon}
\def\ep{\varepsilon}
\def\u{\underline}
\def\c{\centerline}
\def\n{\noindent}
\def\h{\hangindent}


\newcommand{\tcb}{\textcolor{blue}}

% Double spacing is 2, 1.5 spacing is 1.5, ...
\def\baselinestretch{1.0}

% For tables:
\definecolor{Gray}{gray}{0.9}
\newcolumntype{C}{c<{\kern\tabcolsep}@{}}

\begin{document}

\title{Centroids, Clusters, and Crime: Anchoring the Geographic Profiles of Serial Criminals \\(\textbf{Catchy, but informative title})}
\author{Team \# 12345}
\date{\today}

\begin{abstract}
A particularly challenging problem in crime prediction is modeling the behavior of a serial killer. Since finding associations between the victims is difficult, we predict where the criminal will strike next, instead of whom. Such predicting of a criminal's spatial patterns is called geographic profiling.

Research shows that most violent serial criminals tend to commit crimes in a radial band around a central point: home, workplace, or other area of significance to the criminal's activities (for example, a part of town where prostitutes abound). These ``anchor points'' provide the basis for our model.

We assume that the entire domain of analysis is a potential crime spot, movement of the criminal is uninhibited, and the area in question is large enough to contain all possible strike points. We consider the domain a metric space on which predictive algorithms create spatial likelihoods. Additionally, we assume that the offender is a violent serial criminal, since research suggests that serial burglars and arsonists are less likely to follow spatial patterns.

There are substantial differences between one anchor point and several. We treat the single-anchor-point case first, taking the spatial coordinates of the criminal's last strikes and the sequence of the crimes as inputs. Estimating the point to be the centroid of the previous crimes, we generate a ``likelihood crater,'' where height corresponds to the likelihood of a future crime at that location. For the multiple-anchor-point case, we use a cluster-finding and sorting method: We identify groupings in the data and build a likelihood crater around the centroid of each. Each cluster is given weight according to recency and number of points. We test single point vs. multiple points by using the previous crimes to predict the most recent one and comparing with its actual location.
We extract seven datasets from published research. We use four of the datasets in developing our model and examining its response to changes in sequence, geographic concentration, and total number of points. Then we evaluate our models by running blind on the remaining three datasets.
The results show a clear superiority for multiple anchor points.

\textbf{This is the most important section of the paper. Make it beautiful.} The above was the summary page of a winning entry by Damle, West, and Benzel at CU-Boulder in 2010.
\end{abstract}
\maketitle

\newpage

\section{Introduction}\label{Introduction}

\section{Background}\label{Background}


Cite, cite, cite your references. Winning papers have 10-15 references at least. 

\section{Assumptions}\label{Assumptions}

Bullet points can improve readability dramatically. To summarize the above, our assumptions are 

\begin{itemize}
\item The earth is spherical.
\item The earth's rotation is not important on the time scale relevant to the dynamics.
\item The human population on the earth is uniformly distributed. 
\end{itemize}

Or, perhaps you like to number your assumptions to be able to refer to them, later, 

\begin{enumerate}
\item One,
\item Two,
\item Three. 
\end{enumerate}

The paper is organized as follows. The equations of motion are presented and written in terms of a helical coordinate system in \S\ref{Model}. The numerical method used to study the dynamics of an arbitrary (e.g. large amplitude) body shape is described in \S\ref{Numerics}, which provides a basis for comparison for the asymptotic results. In \S\ref{Results}, the predictions of the mathematical model are compared to the results of our numerical method, where we find that things are great. Sometimes. An improved model is described in \S\ref{Something}. We conclude with a discussion in \S\ref{Discussion}, where we discuss strengths and weaknesses of the model, and suggest future directions and stuff.


\section{The mathematical model}\label{Model}

About now, you'll be wanting to write some equations, either in line, $U(\b{x},t)=(\sqrt{4\pi}/M)\b{x}e^t$ (note the use of the latex shortcut defined in the preamble for bold lettering). But for clarity, sometimes you will want to write something bigger, or something small that you want to stand out. After some algebraic manipulation of the expression above, we find
\begin{gather}\label{F}
\mathcal{F}(x;\phi)=\int_0^{x^2}\frac{\xi}{\xi+\sin(\phi)} \, d\xi.
\end{gather}
Note that we have labeled the equation, so that in our prose we can once again say things many pages later like ``recall the expression for $\mathcal{F}(x)$ from Eq.~\eqref{F},'' so that later if we add a new equation we don't need to go and renumber everything. Another thing to do, is refer to whole sections in this manner. According to the assumptions described in \S \ref{Assumptions}, then something. Oh, that brings up another important point. 

\section{Numerical method}\label{Numerics}

Describe exactly what you're going to compute, and how.

\section{Results and analysis}\label{Results}

Results will come in the form of graphs and discussion, and sometimes tables. Table ~\ref{Convergence} shows a nice template. Again, reference the label, not the number. 

\newcolumntype{g}{>{\columncolor{Gray}}l}
\begin{table}[h]
\label{Convergence}
\centering
\begin{tabular}{@{}CCcCCcCCc@{}}
\toprule
\makebox[.3in]{$M$} & \makebox[.8in]{$U$} & \makebox[.8in]{$L$} &  \makebox[.5in]{$N_A/N_W$} & \makebox[.8in]{$U$} & \makebox[.8in]{$L$}& \makebox[.5in]{$N_W$} & \makebox[.8in]{$U$} & \makebox[.8in]{$L$} \\
\colrule
\rowcolor[gray]{0.9} 16 & 0.081733 & 19.60312 & 12 & 0.078402 & 19.59587 &  6 & 0.082795 & 19.59965 \\
32 & 0.081090 & 19.57638 &  24 & 0.081149 & 19.61341 & 12&0.084734 & 19.58195\\
\rowcolor[gray]{0.9} 64 & 0.081018 & 19.57598 &   48 & 0.080930 & 19.60641  & 24&0.085422 & 19.57128\\
128 & 0.081019 & 19.57599 & 96 & 0.080938 & 19.60708 & 48&0.085432 & 19.56992\\
\botrule
\end{tabular}
\caption{Three convergence studies: (1) varying $M$, with $N_{W}=5$ and $N_A/N_{W}=16$ fixed; (2) varying $N_A$, with $N_{W}=5$ and $M=16$ fixed; (3) varying $N_{W}$, with $N_A/N_{W}=16$ and $M=16$ fixed.}
\end{table}



\section{Something more}\label{Something}

A new and improved model? Measuring the robustness of the model? Something? 

\section{Discussion}\label{Discussion}

Summarize the above. Put a bow around your wonderful story. You should have had a beginning and middle, and here is the end. 

\subsection{Strengths of the model}

Maybe some bullet points. 

\subsection{Weaknesses of the model}

\subsection{Future directions}

And conclude with a final summary. 

\appendix

\section{Details for the derivation of $\mathcal{F}(x;\phi)$ in Eq.~\eqref{F}}\label{AppendixA}

A slightly messy calculation that was still small enough to include, but long enough not to want to include up top.

\begin{thebibliography}{10}
\bibitem{nd97} Nelson, Timothy, and Dengler, Nancy. ``Leaf vascular pattern formation.'' The Plant Cell 9.7 (1997): 1121.
\end{thebibliography}

\end{document}

